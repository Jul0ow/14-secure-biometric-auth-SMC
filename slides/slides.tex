\documentclass[9pt]{beamer}

\usepackage[french]{babel}
\usepackage{svg}
\usetheme[progressbar=frametitle]{metropolis}
\usepackage{appendixnumberbeamer}
\usepackage{listings}
\usepackage{csquotes}
\usepackage{xurl}
\usepackage{hyperref}

\usepackage[T1]{fontenc}

\usefonttheme{default}
\usepackage{palatino}
\usepackage[style=ieee, backend=biber]{biblatex}
\setbeamertemplate{bibliography item}[triangle]

\DeclareBibliographyCategory{cited}
\AtEveryCitekey{\addtocategory{cited}{\thefield{entrykey}}}

\usepackage[default]{opensans}
\useinnertheme{rectangles}
\setbeamertemplate{caption}[numbered]

\addbibresource{Bib.bib}
\nocite{*}

\usepackage{datetime}
\newdate{date}{17}{04}{2024}

\date[\displaydate{date}]{}

%\titlegraphic{%
%    \centering
%    \includegraphics[width=0.8\textwidth]{imgs/logoURANUSfull.png}
%}

\makeatletter
\setbeamertemplate{title page}{
  \begin{minipage}[b][\paperheight]{\textwidth}
    \vfill%
    \ifx\inserttitle\@empty\else\usebeamertemplate*{title}\fi
    \ifx\insertsubtitle\@empty\else\usebeamertemplate*{subtitle}\fi
    \usebeamertemplate*{title separator}
    \ifx\beamer@shortauthor\@empty\else\usebeamertemplate*{author}\fi
    \usebeamertemplate*{date}
    \ifx\insertinstitute\@empty\else\usebeamertemplate*{institute}\fi
    \vfill
    \ifx\inserttitlegraphic\@empty\else\inserttitlegraphic\fi
    \vspace*{1cm}
  \end{minipage}
}
\makeatother

\title[LT]{Authentification biométrique par calcul multipartite sécurisé}
\author{Matthieu Colin | Alexandra Delin | Jules Diaz | Maëlys Rimbert}
\date{July 2024}
\institute{EPITA}

\definecolor{codegreen}{rgb}{0,0.6,0}
\definecolor{codegray}{rgb}{0.75,0.75,0.75}
\definecolor{codepurple}{rgb}{0.58,0,0.82}
\definecolor{backcolour}{rgb}{0.95,0.95,0.92}
\lstdefinestyle{mystyle}{
    backgroundcolor=\color{backcolour},   
    commentstyle=\color{codegreen},
    keywordstyle=\color{magenta},
    numberstyle=\tiny\color{codegray},
    stringstyle=\color{codepurple},
    basicstyle=\ttfamily\footnotesize,
    breakatwhitespace=false,         
    breaklines=true,                 
    captionpos=b,                    
    keepspaces=true,                 
    numbers=left,                    
    numbersep=5pt,                  
    showspaces=false,                
    showstringspaces=false,
    showtabs=false,                  
    tabsize=2
}
\lstset{style=mystyle}

\begin{document}

\begin{frame}
	\titlepage
\end{frame}

\begin{frame}{Introduction}
    \begin{itemize}
        \item Deux participants :
        \begin{enumerate}
            \item un serveur ayant une base de données de visages autorisés
            \item un client fournissant les données biométriques de son visage
        \end{enumerate}
        \item Objectif : déterminer si le client est connu par la base de données
        \item Solution en Python car il existe un large choix de bibliothèques de reconnaissance faciale
    \end{itemize}
    \begin{figure}
        \centering
        \includegraphics[scale=0.08]{images_slides/Python-logo-notext.svg.png}
        \caption{Python [\ref{src:python-logo}]}
        \label{fig:enter-label}
    \end{figure}
\end{frame}

\begin{frame}{Face Recognition vs Deepface}
\begin{columns}[T] % align columns
    \begin{column}{.48\textwidth}
        \begin{center}
            \begin{figure}
            \centering
            \includegraphics[scale=0.3]{images_slides/face_reco.png}
            \caption{Démonstration de face-recognition [\ref{src:face-reco-demo}]}
            \label{fig:enter-label}
            \end{figure}
        \end{center}
    \end{column}%
    \hfill%
    \begin{column}{.48\textwidth}
        \begin{center}
            \begin{figure}
            \centering
            \includegraphics[scale=0.08]{images_slides/deepface.png}
            \caption{Logo Deepface [\ref{src:deepface-logo}]}
            \label{fig:enter-label}
            \end{figure}
        \end{center}
    \end{column}%
    \end{columns}
\begin{itemize}
    \item Bibliothèques Python de reconnaissance faciale
    \item Fonctionnalités similaires, dont l'extraction d'\textit{embeddings} correspondant à des visages
    \item \textit{Embeddings} de 128 et 4096 éléments
    \item Problèmes de performance avec Deepface : nos machines n'étaient pas assez puissantes pour supporter les calculs sur les vecteurs
\end{itemize}
Nous avons choisi de travailler avec \textbf{face-recognition} afin de simplifier les opérations et de réduire le temps de calcul.
    
\end{frame}

\begin{frame}{Calcul Multipartite Sécurisé}
    
\end{frame}

% Jules
\begin{frame}{Partage de clé secrète de Shamir}
    \begin{itemize}[<+- | alert@+>]
        \item \textit{Shamir's Secret Sharing}
        \begin{itemize}
            \item Permet de \textbf{scinder un secret en plusieurs morceaux} appelées \textit{parties}
            \item Chaque \textit{partie} ne donne \textbf{aucune information} sur le secret en lui même
            \item L'algorithme permet de définir un seuil indiquant le nombre de \textit{parties} minimum afin de pouvoir reconstituer le secret 
        \end{itemize}
        \item Principe simple basé sur :
        \begin{itemize}
            \item Les propriétés des polynômes
            \item L'interpolation de Lagrange
        \end{itemize}
        \item Opération possible sur les parties
        \begin{itemize}
            \item Addition
            \item Multiplication par un scalaire
            \item Multiplication
            \item \ldots
        \end{itemize}
    \end{itemize}
\end{frame}

\begin{frame}{Exemple - Seuil à deux}
    \begin{itemize}
        \item Prenons le secret $s = -1$
        \item Formons le polynôme $q(x) = 2x - 1$
        \begin{itemize}
            \item $2$ un coefficient aléatoire
            \item $-1$ notre secret
        \end{itemize}
        \item Exemple de points partageables :
        \begin{itemize}
            \item (1, 1)
            \item (2, 3)
        \end{itemize}
    \end{itemize}
\end{frame}

\begin{frame}{Exemple - Visualisation}
    \begin{figure}[ht]
        \begin{minipage}[b]{0.45\linewidth}
            \centering
            \includegraphics[width=\textwidth]{images/infinite_sol.png}
            \caption{Infinité de solutions}
            \label{fig:infinite-sol}
        \end{minipage}
        \hspace{0.5cm}
        \pause
        \begin{minipage}[b]{0.45\linewidth}
            \centering
            \includegraphics[width=\textwidth]{images/unique_sol.png}
            \caption{Solution unique}
            \label{fig:b}
        \end{minipage}
    \end{figure}
\end{frame}

\begin{frame}{Exemple - Construction du polynôme}
    \begin{itemize}
        \item <1-> Interpolation Lagrangienne :\\
        \[ L(X)=\sum_{j=0}^{n}y_i(\prod_{j=0,i\neq j}^n \frac{X - x_i}{x_j - x_i}) \]\\
        \[ q(x) = 1 * \frac{x-2}{1-2} + 3 * \frac{x - 1}{2 - 1} =  2x - 1\]
        \item <2-> Récupération du secret :\\
        \[ q(0) = -1 \]
    \end{itemize}
\end{frame}

\begin{frame}{MPyC}
    \begin{itemize}[<+- | alert@+>]
        \item \textit{Multiparty Computation in Python}
        \begin{itemize}
            \item Panel considérable de fonctionnalités
            \item Simple d'utilisation 
            \item Exemples clairs
            \item Bibliothèque maintenue
        \end{itemize}
        \item S'articule facilement avec les bibliothèques de reconnaissances faciales
    \end{itemize}
\end{frame}

\begin{frame}{Expérimentation \& Résultat}
    \textbf{7} scripts Python
    \begin{itemize}
        \item Distance euclidienne entre deux images utilisant le \textbf{SMC}
        \item Script d'évaluation de performances de l'algorithme MPyC
        \item Extraction des \textit{faces encodings} d'un jeu d'image
        \item Comparaison entre la bibliothèque \textit{DeepFace} et \textit{Face Recognition}
        \item Démonstrateur (Autorisation par reconnaissance faciale)
        \begin{itemize}
            \item Serveur
            \item Client
        \end{itemize}
    \end{itemize}
\end{frame}

\begin{frame}[fragile]{Expérimentation \& Résultat}
    Calcule de la distance euclidienne avec la bibliothèque MPyC
    \begin{scriptsize}
        \begin{lstlisting}[showstringspaces=false, language=python]
    embedding = secfpx.array(face_encoding)

    # Recuperation de l'image de la seconde partie
    user, server = mpc.input(embedding)

    # print('Computing the distance')
    distance = np.subtract(user, server)
    # print('Computing the euclidian distance')
    # print('Multiply')
    euclidian = np.multiply(distance, distance)
    # print('Sum')
    euclidian = np.sum(euclidian)

    # print('Printing the result')
    euclidian = await mpc.output(euclidian)
    # print('Sqrt')
    euclidian = np.sqrt(euclidian)
    # print('Result', euclidian)
    return euclidian
        \end{lstlisting}
    \end{scriptsize}
\end{frame}

\begin{frame}{Évaluation des performances}
    Il faut moins de \textbf{30 ms} pour vérifier si deux images contiennent la même personne.

    Nous avons déterminé que la valeur de seuil la plus optimal était proche de \textbf{0,575}
    \begin{figure}[H]
        \centering
        \includegraphics[scale=0.4]{images/sucess_rate_evolution.png}
        \caption{Courbe d'évolution du taux de succès par valeur seuil}
        \label{fig:sucess_rate_evolution}
    \end{figure}
\end{frame}

\begin{frame}{Défis rencontrés}

    \begin{itemize}
        \item Recherche de bibliothèques
        \item Documentations peu claires et complexes
        \item Calculs très consommateurs de ressources
        \item Difficultés pour intégrer les fonctions des bibliothèques à notre code
    \end{itemize}
    
\end{frame}

\begin{frame}{Mesures futures}
    \begin{itemize}
        \item Potentiellement utilisable sur smartphone
        \item Solution sécurisée
        \item Processus lent à optimiser
        \item Scalabilité limitée
        \item Pas utilisable en l'état dans des grandes entreprises
        \item Il reste néanmoins possible de développer une solution plus rapide, fiable et sécurisée
    \end{itemize}
\end{frame}

\begin{frame}{Conclusion}
    \begin{itemize}
        \item Un sujet très complexe permettant la protection des données
        \item De multiples applications dans la vraie vie
        \item Une preuve de concept fonctionnelle
        \item Chiffrement homomorphe
    \end{itemize}
\end{frame}

\begin{frame}[allowframebreaks]
    \frametitle{Sources}
    \begin{itemize}
        \label{src:face-reco-demo}\item Démonstration de face-recognition : \url{https://pypi.org/project/face-recognition/}
        \label{src:deepface-logo}\item Logo Deepface : \url{https://pypi.org/project/deepface/}
        \item \label{src:python-logo} Logo Python : \url{https://fr.m.wikipedia.org/wiki/Fichier:Python-logo-notext.svg}
    \end{itemize}
\end{frame}

\printbibliography

\end{document}